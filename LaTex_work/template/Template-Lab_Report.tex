\documentclass{ctexrep}
\usepackage[utf8]{inputenc}
\usepackage{fullpage}
\usepackage{times}
\usepackage{fancyhdr,graphicx,amsmath,amssymb}
\usepackage[linesnumbered,lined,boxed,commentsnumbered, ruled, vlined]{algorithm2e}

%\pagestyle{headings}
%\usepackage{fancyhdr}
%\pagestyle{fancy}
%\fancyhf{}
%\rhead{计算机科学与技术学院}
%\lhead{华中科技大学}

\begin{titlepage}
\title{CS15210 Lab}
\author{Yuzhe SHI}
\date{December 2019}
\end{titlepage}

%%% begin of entire document
\begin{document}
%%%
\maketitle
%%%
\tableofcontents
%%% catalogue

%%% global abstract
\begin{abstract}
    这是整个实验课程的总摘要。
\end{abstract}
%%%

%%% introduction
\chapter{简介}
这是关于实验内容的总介绍。
%%%

%%% prologue
\chapter{理论基础回顾}
这里回顾实验课程中涉及到的理论基础。
%%% PRAM Model
\section{PRAM模型:并行程序的抽象}

%%%

%%% Complexity analyzing 
\section{Work $and$ Span: 算法复杂度分析方法}
\subsection{Work $and$ Span}

%%%

%%% Provation of Algorithms
\section{算法正确性的证明策略}

%%%

%%% Abstract data types
\section{三种抽象数据类型}
\subsection{Sequence: 顺序结构}
\subsection{Binary Search Tree: 平衡搜索树结构}
\subsection{Sets $and$ Table: 集合与表}

%%% Lab 1
\chapter{实验1:快速排序}
%%% introduction
\section{实验介绍}
简要介绍这个实验。
%%%

%%% describe the problem of the lab
\section{问题描述}
用自然语言描述实验题目要求。
%%%

%%% algorithm description 
\section{算法}
%%% describe the algorithm with nature language
\subsection{算法描述}
用自然语言和伪代码描述算法。
%%%
%%% pseudo code
\begin{algorithm}
\SetAlgoLined
\KwData{data}
\KwResult{Write here the result }
 initialization\;
 \While{While condition}{
  instructions\;
  \eIf{condition}{
   instructions1\;
   instructions2\;
   }{
   instructions3\;
  }
 }
 \caption{How to write algorithms}
\end{algorithm}
%%%

%%% asymptotically analysing  
\subsection{算法分析}
使用渐进分析法分析算法的复杂度。
%%%

%%% experiments
\section{实验}
%%% select a set of example and describe how the algorithm works with the example
\subsection{样例分析}
选择一组样例,描述算法在这组样例上的表现。
%%%
%%% institute the way to deal with extreme cases
\subsection{边界条件处理}
介绍算法如何处理极端边界条件。
%%%

%%% conclusion
\section{总结}
总结这个算法。
%%%
%%% end of Lab 1

%%% Lab 2
\chapter{实验2:最短路径}
%%% introduction
\section{实验介绍}
简要介绍这个实验。
%%%

%%% describe the problem of the lab
\section{问题描述}
用自然语言描述实验题目要求。
%%%

%%% algorithm description 
\section{算法}
%%% describe the algorithm with nature language
\subsection{算法描述}
用自然语言和伪代码描述算法。
%%%
%%% pseudo code
\begin{algorithm}
\SetAlgoLined
\KwData{data}
\KwResult{Write here the result }
 initialization\;
 \While{While condition}{
  instructions\;
  \eIf{condition}{
   instructions1\;
   instructions2\;
   }{
   instructions3\;
  }
 }
 \caption{How to write algorithms}
\end{algorithm}
%%%

%%% asymptotically analysing  
\subsection{算法分析}
使用渐进分析法分析算法的复杂度。
%%%

%%% experiments
\section{实验}
%%% select a set of example and describe how the algorithm works with the example
\subsection{样例分析}
选择一组样例,描述算法在这组样例上的表现。
%%%
%%% institute the way to deal with extreme cases
\subsection{边界条件处理}
介绍算法如何处理极端边界条件。
%%%

%%% conclusion
\section{总结}
总结这个算法。
%%%
%%% end of Lab 2

%%% Lab 3
\chapter{实验3:最大括号长度}
%%% introduction
\section{实验介绍}
简要介绍这个实验。
%%%

%%% describe the problem of the lab
\section{问题描述}
用自然语言描述实验题目要求。
%%%

%%% algorithm description 
\section{算法}
%%% describe the algorithm with nature language
\subsection{算法描述}
用自然语言和伪代码描述算法。
%%%
%%% pseudo code
\begin{algorithm}
\SetAlgoLined
\KwData{data}
\KwResult{Write here the result }
 initialization\;
 \While{While condition}{
  instructions\;
  \eIf{condition}{
   instructions1\;
   instructions2\;
   }{
   instructions3\;
  }
 }
 \caption{How to write algorithms}
\end{algorithm}
%%%

%%% asymptotically analysing  
\subsection{算法分析}
使用渐进分析法分析算法的复杂度。
%%%

%%% experiments
\section{实验}
%%% select a set of example and describe how the algorithm works with the example
\subsection{样例分析}
选择一组样例,描述算法在这组样例上的表现。
%%%
%%% institute the way to deal with extreme cases
\subsection{边界条件处理}
介绍算法如何处理极端边界条件。
%%%

%%% conclusion
\section{总结}
总结这个算法。
%%%
%%% end of Lab 3

%%% Lab 4
\chapter{实验4:天际线}
%%% introduction
\section{实验介绍}
简要介绍这个实验。
%%%

%%% describe the problem of the lab
\section{问题描述}
用自然语言描述实验题目要求。
%%%

%%% algorithm description 
\section{算法}
%%% describe the algorithm with nature language
\subsection{算法描述}
用自然语言和伪代码描述算法。
%%%
%%% pseudo code
\begin{algorithm}
\SetAlgoLined
\KwData{data}
\KwResult{Write here the result }
 initialization\;
 \While{While condition}{
  instructions\;
  \eIf{condition}{
   instructions1\;
   instructions2\;
   }{
   instructions3\;
  }
 }
 \caption{How to write algorithms}
\end{algorithm}
%%%

%%% asymptotically analysing  
\subsection{算法分析}
使用渐进分析法分析算法的复杂度。
%%%

%%% experiments
\section{实验}
%%% select a set of example and describe how the algorithm works with the example
\subsection{样例分析}
选择一组样例,描述算法在这组样例上的表现。
%%%
%%% institute the way to deal with extreme cases
\subsection{边界条件处理}
介绍算法如何处理极端边界条件。
%%%

%%% conclusion
\section{总结}
总结这个算法。
%%%
%%% end of Lab 4

%%% Lab 1
\chapter{实验5:括号匹配}
%%% introduction
\section{实验介绍}
简要介绍这个实验。
%%%

%%% describe the problem of the lab
\section{问题描述}
用自然语言描述实验题目要求。
%%%

%%% algorithm description 
\section{算法}
%%% describe the algorithm with nature language
\subsection{算法描述}
用自然语言和伪代码描述算法。
%%%
%%% pseudo code
\begin{algorithm}
\SetAlgoLined
\KwData{data}
\KwResult{Write here the result }
 initialization\;
 \While{While condition}{
  instructions\;
  \eIf{condition}{
   instructions1\;
   instructions2\;
   }{
   instructions3\;
  }
 }
 \caption{How to write algorithms}
\end{algorithm}
%%%

%%% asymptotically analysing  
\subsection{算法分析}
使用渐进分析法分析算法的复杂度。
%%%

%%% experiments
\section{实验}
%%% select a set of example and describe how the algorithm works with the example
\subsection{样例分析}
选择一组样例,描述算法在这组样例上的表现。
%%%
%%% institute the way to deal with extreme cases
\subsection{边界条件处理}
介绍算法如何处理极端边界条件。
%%%

%%% conclusion
\section{总结}
总结这个算法。
%%%
%%% end of Lab 5

%%% Lab 6
\chapter{实验6:高精度运算}
%%% introduction
\section{实验介绍}
简要介绍这个实验。
%%%

%%% describe the problem of the lab
\section{问题描述}
用自然语言描述实验题目要求。
%%%

%%% algorithm description 
\section{算法}
%%% describe the algorithm with nature language
\subsection{算法描述}
用自然语言和伪代码描述算法。
%%%
%%% pseudo code
\begin{algorithm}
\SetAlgoLined
\KwData{data}
\KwResult{Write here the result }
 initialization\;
 \While{While condition}{
  instructions\;
  \eIf{condition}{
   instructions1\;
   instructions2\;
   }{
   instructions3\;
  }
 }
 \caption{How to write algorithms}
\end{algorithm}
%%%

%%% asymptotically analysing  
\subsection{算法分析}
使用渐进分析法分析算法的复杂度。
%%%

%%% experiments
\section{实验}
%%% select a set of example and describe how the algorithm works with the example
\subsection{样例分析}
选择一组样例,描述算法在这组样例上的表现。
%%%
%%% institute the way to deal with extreme cases
\subsection{边界条件处理}
介绍算法如何处理极端边界条件。
%%%

%%% conclusion
\section{总结}
总结这个算法。
%%%
%%% end of Lab 6

%%% Lab 7
\chapter{实验7:割点与割边的判定}
%%% introduction
\section{实验介绍}
简要介绍这个实验。
%%%

%%% describe the problem of the lab
\section{问题描述}
用自然语言描述实验题目要求。
%%%

%%% algorithm description 
\section{算法}
%%% describe the algorithm with nature language
\subsection{算法描述}
用自然语言和伪代码描述算法。
%%%
%%% pseudo code
\begin{algorithm}
\SetAlgoLined
\KwData{data}
\KwResult{Write here the result }
 initialization\;
 \While{While condition}{
  instructions\;
  \eIf{condition}{
   instructions1\;
   instructions2\;
   }{
   instructions3\;
  }
 }
 \caption{How to write algorithms}
\end{algorithm}
%%%

%%% asymptotically analysing  
\subsection{算法分析}
使用渐进分析法分析算法的复杂度。
%%%

%%% experiments
\section{实验}
%%% select a set of example and describe how the algorithm works with the example
\subsection{样例分析}
选择一组样例,描述算法在这组样例上的表现。
%%%
%%% institute the way to deal with extreme cases
\subsection{边界条件处理}
介绍算法如何处理极端边界条件。
%%%

%%% conclusion
\section{总结}
总结这个算法。
%%%
%%% end of Lab 7

%%% Lab 8
\chapter{实验8:静态区间查询}
%%% introduction
\section{实验介绍}
简要介绍这个实验。
%%%

%%% describe the problem of the lab
\section{问题描述}
用自然语言描述实验题目要求。
%%%

%%% algorithm description 
\section{算法}
%%% describe the algorithm with nature language
\subsection{算法描述}
用自然语言和伪代码描述算法。
%%%
%%% pseudo code
\begin{algorithm}
\SetAlgoLined
\KwData{data}
\KwResult{Write here the result }
 initialization\;
 \While{While condition}{
  instructions\;
  \eIf{condition}{
   instructions1\;
   instructions2\;
   }{
   instructions3\;
  }
 }
 \caption{How to write algorithms}
\end{algorithm}
%%%

%%% asymptotically analysing  
\subsection{算法分析}
使用渐进分析法分析算法的复杂度。
%%%

%%% experiments
\section{实验}
%%% select a set of example and describe how the algorithm works with the example
\subsection{样例分析}
选择一组样例,描述算法在这组样例上的表现。
%%%
%%% institute the way to deal with extreme cases
\subsection{边界条件处理}
介绍算法如何处理极端边界条件。
%%%

%%% conclusion
\section{总结}
总结这个算法。
%%%
%%% end of Lab 8

%%% Lab 9
\chapter{实验9:素性判定}
%%% introduction
\section{实验介绍}
简要介绍这个实验。
%%%

%%% describe the problem of the lab
\section{问题描述}
用自然语言描述实验题目要求。
%%%

%%% algorithm description 
\section{算法}
%%% describe the algorithm with nature language
\subsection{算法描述}
用自然语言和伪代码描述算法。
%%%
%%% pseudo code
\begin{algorithm}
\SetAlgoLined
\KwData{data}
\KwResult{Write here the result }
 initialization\;
 \While{While condition}{
  instructions\;
  \eIf{condition}{
   instructions1\;
   instructions2\;
   }{
   instructions3\;
  }
 }
 \caption{How to write algorithms}
\end{algorithm}
%%%

%%% asymptotically analysing  
\subsection{算法分析}
使用渐进分析法分析算法的复杂度。
%%%

%%% experiments
\section{实验}
%%% select a set of example and describe how the algorithm works with the example
\subsection{样例分析}
选择一组样例,描述算法在这组样例上的表现。
%%%
%%% institute the way to deal with extreme cases
\subsection{边界条件处理}
介绍算法如何处理极端边界条件。
%%%

%%% conclusion
\section{总结}
总结这个算法。
%%%
%%% end of Lab 9

\chapter{实验课总结与心得体会}


%%% end of entire document
\end{document}
%%%